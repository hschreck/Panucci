\documentclass{article}
\usepackage[utf8]{inputenc}
\usepackage[english]{babel}
\usepackage{outlines}
\setlength{\parindent}{4em}
\setlength{\parskip}{1em}
\begin{document}
\title{Using Panucci}
\maketitle
\section{What is Panucci}


Panucci is a network-based system testing and imaging platform.\linebreak\linebreak
Panucci is a collection of multiple open-source and free software packages, custom built for our particular use case - multiple images going onto multiple computers at once.\pagebreak

\section{Installing Panucci}
The simplest way to set up Panucci is to use Ubuntu Server and DRBL (Diskless Remote Boot for Linux).  A standard PXE environment is also viable, but is outside the scope of this document.  If you will be installing Panucci on a standard PXE environment, skip to the appropriate section.
\subsection{Overview of requirements}
\begin{outline}
  \1 PXE Server
    \2 Tested using Ubuntu Server and DRLB (Diskless Remote Boot for Linux)
    \2 Any PXE server will work
  \1 NFS server (for images)\textbf{Note}: Using DRBL will place the NFS server on the same machine as the PXE server.
  \1 Software required on the PXE image
    \2 Linux
    \2 Ruby
      \3 Sinatra gem
      \3 dotenv gem
    \2 A web browser
      \3 Chromium (tested)
      \3 Firefox (untested, but should work)
    \2 memtester
    \2 Clonezilla
    \2 NFS (for images)
    \2 dmidecode
    \2 sudo
    \2 smartmontools (specifically, \textit{smartctl})
    \2 seeker (a small utility used to test seek time on drive)
    \2 i3 Window Manager (this can be swapped out)
    \2 xterm (this can be swapped out)
    \2 rerun (this can be left out, follow notes for that case)
\end{outline}\pagebreak

\section{Hardware}
The hardware requirements for Panucci are meager, but better hardware, especially networking and storage, will greatly improve the performance of the platform.  While there are no defined minimums, standard server hardware serves as a solid starting point.
\subsection{Storage}
Many computer images require between 20 and 40GB of storage per image.  While Panucci is able to handle multiple images for a single device line with little overhead over the original image, the storage requirements can quickly become cumbersome without proper planning.

\subsubsection{Storage Space}
To calculate the amount of storage needed to comfortably serve your entire line, calculate the maximum expected size of each image (in this example, 40GB) and multiply it by the number of product lines you expect to carry.
\begin{center}
40GB * 40 lines = \textbf{1.6TB} of total storage
\end{center}
As an added buffer, it is recommended to add an additional 25\% to the pool to account for new product lines in the future.
\subsubsection{Storage Speed}
Storage speed requirements will vary based on the number of devices you will be imaging simultaneously.
To determine speed requirement, multiply the number of machines you expect to image simultaneously by the expected ``pull speed".  If you will be using gigabit network connections (which is the limit of most current hardware), figure on approximately 115MB/s per unit.
\begin{center}
  16 simultaneous units * 115MB/s per unit = \textbf{1840MB/s}
\end{center}
This number assumes that all units will be ``pulling" at their network-limited speed of 115MB/s, which many devices simply will not be capable of, so there is already some degree of building for growth included.\linebreak
Additionally, the calculated number is the theoretical threshold to have all units imaging at full speed.  If a lower speed to image is acceptable, lowering the access speed is a valid step to reduce implementation costs.\linebreak\linebreak
\subsubsection{Recommendations}
There are some standard methods to help improve your storage for Panucci.
\begin{itemize}
  \item Consider using RAID (especially RAID 1+0).  Properly configured, RAID can offer significant performance and resiliency advantages.  If hardware RAID is not available, LVM or software RAID are viable alternatives, but you will see a performance hit from their use.
  \item Consider splitting out the drives by use.  In a multi-array or multi-drive environment, place machine images onto one array or drive (usually as \textit{/home/partimag}) and place the core operating system on another array or drive.
  \item Consider using Solid State Drives (SSDs) or SAS drives.  Both provide a performance gain over traditional SATA drives.  SSDs are more expensive per unit of storage, but are usually the fastest option available.  SAS drives offer more storage, but are slower.
\end{itemize}
\subsection{Network}
Given that Panucci is a network-based imaging platform, it makes sense that it is heavily network-bound.
\subsubsection{Network Ports}
Ideally, each client machine will operate at the full speed of the network connection (although this is not always the case due to a number of factors).  This can simplify the network calculations if both the server and client are running the same speed (e.g., both are using gigabit ethernet).  In such instances, you will need as many network ports on the server as you wish to have simultaneous clients.

\pagebreak
\section{Installation Procedure}

\end{document}
